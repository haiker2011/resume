%%%%%%%%%%%%%%%%%%%%%%%%%%%%%%%%%%%%%%%
% Deedy - One Page Two Column Resume
% LaTeX Template
% Version 1.2 (16/9/2014)
%
% Original author:
% Debarghya Das (http://debarghyadas.com)
%
% Original repository:
% https://github.com/deedydas/Deedy-Resume
%
% IMPORTANT: THIS TEMPLATE NEEDS TO BE COMPILED WITH XeLaTeX
%
% This template uses several fonts not included with Windows/Linux by
% default. If you get compilation errors saying a font is missing, find the line
% on which the font is used and either change it to a font included with your
% operating system or comment the line out to use the default font.
% 
%%%%%%%%%%%%%%%%%%%%%%%%%%%%%%%%%%%%%%
% 
% TODO:
% 1. Integrate biber/bibtex for article citation under publications.
% 2. Figure out a smoother way for the document to flow onto the next page.
% 3. Add styling information for a "Projects/Hacks" section.
% 4. Add location/address information
% 5. Merge OpenFont and MacFonts as a single sty with options.
% 
%%%%%%%%%%%%%%%%%%%%%%%%%%%%%%%%%%%%%%
%
% CHANGELOG:
% v1.1:
% 1. Fixed several compilation bugs with \renewcommand
% 2. Got Open-source fonts (Windows/Linux support)
% 3. Added Last Updated
% 4. Move Title styling into .sty
% 5. Commented .sty file.
%
%%%%%%%%%%%%%%%%%%%%%%%%%%%%%%%%%%%%%%%
%
% Known Issues:
% 1. Overflows onto second page if any column's contents are more than the
% vertical limit
% 2. Hacky space on the first bullet point on the second column.
%
%%%%%%%%%%%%%%%%%%%%%%%%%%%%%%%%%%%%%%


\documentclass[]{deedy-resume-openfont}
\usepackage{fancyhdr}
    
\pagestyle{fancy}
\fancyhf{}
    
\begin{document}

%%%%%%%%%%%%%%%%%%%%%%%%%%%%%%%%%%%%%%
%
%     LAST UPDATED DATE
%
%%%%%%%%%%%%%%%%%%%%%%%%%%%%%%%%%%%%%%
\lastupdated

%%%%%%%%%%%%%%%%%%%%%%%%%%%%%%%%%%%%%%
%
%     TITLE NAME
%
%%%%%%%%%%%%%%%%%%%%%%%%%%%%%%%%%%%%%%
\namesection{孙}{海洲}{ \urlstyle{same}\href{haizhou.uestc2011@gmail.com}{sunhaizhou.ai@gmail.com} | 1851 0343 050 | {寻找 2019 云原生开发/后端/AI infra 实习或者全职工作}
}

%%%%%%%%%%%%%%%%%%%%%%%%%%%%%%%%%%%%%%
%
%     COLUMN ONE
%
%%%%%%%%%%%%%%%%%%%%%%%%%%%%%%%%%%%%%%

\begin{minipage}[t]{0.3\textwidth} 

%%%%%%%%%%%%%%%%%%%%%%%%%%%%%%%%%%%%%%
%     EDUCATION
%%%%%%%%%%%%%%%%%%%%%%%%%%%%%%%%%%%%%%

\section{教育经历} 
\sectionsep

\subsection{中国科学院大学}
\descript{硕士学位,计算机技术}
\location{2017.09-至今}
\sectionsep

\subsection{电子科技大学}
\descript{学士学位,软件工程}
\location{2011.09-2015.06}
\sectionsep

%%%%%%%%%%%%%%%%%%%%%%%%%%%%%%%%%%%%%%
%     LINKS
%%%%%%%%%%%%%%%%%%%%%%%%%%%%%%%%%%%%%%

\section{相关链接}
\sectionsep    
Github:// \href{https://github.com/haiker2011}{\bf haiker2011} \\
{(\textbf{ 20+ }关注者)} \\

\section{部分博文}
\sectionsep
\href{http://www.servicemesher.com/blog/knative-whittling-down-the-code/}{Knative:精简代码之道} \\
\href{http://www.servicemesher.com/blog/building-a-control-plane-for-envoy/}{如何基于 Envoy 构建一个多用途控制平面} \\


\sectionsep

\section{语言能力}
\sectionsep
英语四级 496 \\
\sectionsep

%%%%%%%%%%%%%%%%%%%%%%%%%%%%%%%%%%%%%%
%     SKILLS
%%%%%%%%%%%%%%%%%%%%%%%%%%%%%%%%%%%%%%

\section{相关技能}
\sectionsep
\subsection{编程}
% \location{超过 5000 行}
% Java   \\
\location{1000 - 5000 行}
Java  \\
\location{低于 1000 行}
Go \textbullet{} Python \textbullet{}  \LaTeX\ \\ 
\sectionsep

\subsection{云计算}
\location{一般}
Docker \textbullet{} Kubernetes   \\
\location{了解}
Istio \textbullet{} Knative  \\
\sectionsep

\subsection{DevOps}
\location{一般}
微服务 \textbullet{} 云原生开发 \textbullet{} Jenkins 
\sectionsep

%%%%%%%%%%%%%%%%%%%%%%%%%%%%%%%%%%%%%%
%     COURSEWORK
%%%%%%%%%%%%%%%%%%%%%%%%%%%%%%%%%%%%%%

% \section{修读课程}
% \subsection{Graduate}
% Advanced Machine Learning \\
% Open Source Software Engineering \\
% Advanced Interactive Graphics \\
% Compilers + Practicum \\
% Cloud Computing \\
% Evolutionary Computation \\
% Defending Computer Networks \\
% Machine Learning \\
% \sectionsep

%%%%%%%%%%%%%%%%%%%%%%%%%%%%%%%%%%%%%%
%
%     COLUMN TWO
%
%%%%%%%%%%%%%%%%%%%%%%%%%%%%%%%%%%%%%%

\end{minipage}
\hfill
\begin{minipage}[t]{0.68\textwidth}

%%%%%%%%%%%%%%%%%%%%%%%%%%%%%%%%%%%%%%
%     EXPERIENCE
%%%%%%%%%%%%%%%%%%%%%%%%%%%%%%%%%%%%%%

\section{工作实习经历}

\sectionsep
\runsubsection{中国科学院计算技术研究所}
\descript{学生}
\location{2017.08-至今 | 北京}
\vspace{\topsep}
\vspace{\topsep}
\begin{tightemize}
\item 关键字、情感分析、命名实体识别、分词、新闻抽取等分布式服务框架重构
\item 分布式服务框架 Docker 化,调研 Docker、Kubernetes 和 Istio 等使用方法
\end{tightemize}
\sectionsep

\sectionsep
\runsubsection{北京数字认证股份有限公司}
\descript{Android 开发工程师}
\location{2015.09-2017.07 | 北京}
\vspace{\topsep}
\vspace{\topsep}
\begin{tightemize}
\item 产品开发:负责 Android 端产品 SDK 需求定义与开发
\item 项目定制:中国人寿、上海银行等项目定制开发
\item 项目实施:负责项目使用 SDK 实施的技术支持
\end{tightemize}
\sectionsep

\sectionsep
\runsubsection{成都尚医信息科技有限公司}
\descript{Android 开发工程师(实习)}
\location{2015.02-2015.07 | 成都}
\vspace{\topsep}
\begin{tightemize}
\item APP 测试:负责测试术康 APP 医生端和患者端测试
\item APP 开发:开发二维码扫描功能
\end{tightemize}
\sectionsep


%%%%%%%%%%%%%%%%%%%%%%%%%%%%%%%%%%%%%%
%     RESEARCH
%%%%%%%%%%%%%%%%%%%%%%%%%%%%%%%%%%%%%%

\section{项目与论文}
\sectionsep

\runsubsection{\href{https://github.com/haiker2011/awesome-nlp-sentiment-analysis}{\bf awesome-nlp-sentiment-analysis}}
\descript{Owner}
\location{2019.3}
\begin{tightemize}
    \item 最新情感分析论文整理,在 GitHub 上获得 \textbf{15 stars}
    \item 论文、代码、数据集方面大家研究
    \end{tightemize}
\sectionsep
% \end{minipage}

% \newpage
% \pagestyle{fancy}
% \fancyhf{}

% % PAGE TWO

% \begin{minipage}[t]{0.3\textwidth}



%%%%%%%%%%%%%%%%%%%%%%%%%%%%%%%%%%%%%%
%
%     COLUMN TWO
%
%%%%%%%%%%%%%%%%%%%%%%%%%%%%%%%%%%%%%%

% \end{minipage} 
% \hfill
% \begin{minipage}[t]{0.68\textwidth} 


%%%%%%%%%%%%%%%%%%%%%%%%%%%%%%%%%%%%%%
%     OPEN SOURCE
%%%%%%%%%%%%%%%%%%%%%%%%%%%%%%%%%%%%%%

%%%%%%%%%%%%%%%%%%%%%%%%%%%%%%%%%%%%%%
%     AWARDS
%%%%%%%%%%%%%%%%%%%%%%%%%%%%%%%%%%%%%%

\runsubsection{\href{https://github.com/servicemesher/getting-started-with-knative}{\bf getting-started-with-knative}}
\descript{servicemesher member}
\location{2019.3}
\begin{tightemize}
    \item 第三章 Build 翻译,在 GitHub 上获得 \textbf{46 stars}
    \item 其他章节 reviewer
    \end{tightemize}
\sectionsep

%%%%%%%%%%%%%%%%%%%%%%%%%%%%%%%%%%%%%%
%     PUBLICATIONS
%%%%%%%%%%%%%%%%%%%%%%%%%%%%%%%%%%%%%%

% \section{Publications} 
% \renewcommand\refname{\vskip -1.5cm} % Couldn't get this working from the .cls file
% \bibliographystyle{abbrv}
% \bibliography{publications}
% \nocite{*}


%%%%%%%%%%%%%%%%%%%%%%%%%%%%%%%%%%%%%%
%     AWARDS
%%%%%%%%%%%%%%%%%%%%%%%%%%%%%%%%%%%%%%

% \section{所获奖项} 
% \begin{tabular}{rll}
% 2013         & 奖学金  & 人民奖学金 \\
% 2014         & 奖学金  & 人民奖学金 \\

% \end{tabular}
% \sectionsep


%%%%%%%%%%%%%%%%%%%%%%%%%%%%%%%%%%%%%%
%     OPEN SOURCE
%%%%%%%%%%%%%%%%%%%%%%%%%%%%%%%%%%%%%%

\section{开源贡献}
\begin{tabular}{ll}
\href{https://github.com/haiker2011/awesome-nlp-sentiment-analysis/commits?author=haiker2011}{\bf haiker2011/awesome-nlp-sentiment-analysis} & 情感分析论文整理,maintainer \\
\href{https://github.com/servicemesher/getting-started-with-knative/commits?author=haiker2011}{\bf servicemesher/getting-started-with-knative} & 《Knative 入门中文版》第三章翻译 \\
\href{https://github.com/istio/istio.io/commits?author=haiker2011}{\bf istio/istio.io} & 翻译最新文档  \\
\href{https://github.com/rootsongjc/awesome-cloud-native/commits?author=haiker2011}{\bf rootsongjc/awesome-cloud-native} & 添加AI模块相关云原生组件 \\
\href{https://github.com/servicemesher/trans/commits?author=haiker2011}{\bf servicemesher/trans} & 翻译文章 Knative:精简代码之道 \\
\href{https://github.com/servicemesher/istio-handbook/commits?author=haiker2011}{\bf servicemesher/istio-handbook} & 《istio hanbook》参与 \\
\href{https://github.com/servicemesher/istio-knowledge-map/commits?author=haiker2011}{\bf servicemesher/istio-knowledge-map} & istio 知识图谱 参与 \\
更多请见 \href{https://github.com/haiker2011}{\bf haiker2011 @ GitHub} & \\
\end{tabular}
\sectionsep


\end{minipage} 
\end{document}  \documentclass[]{article}
