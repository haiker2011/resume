%%%%%%%%%%%%%%%%%%%%%%%%%%%%%%%%%%%%%%%
% Deedy - One Page Two Column Resume
% LaTeX Template
% Version 1.2 (16/9/2014)
%
% Original author:
% Debarghya Das (http://debarghyadas.com)
%
% Original repository:
% https://github.com/deedydas/Deedy-Resume
%
% IMPORTANT: THIS TEMPLATE NEEDS TO BE COMPILED WITH XeLaTeX
%
% This template uses several fonts not included with Windows/Linux by
% default. If you get compilation errors saying a font is missing, find the line
% on which the font is used and either change it to a font included with your
% operating system or comment the line out to use the default font.
% 
%%%%%%%%%%%%%%%%%%%%%%%%%%%%%%%%%%%%%%
% 
% TODO:
% 1. Integrate biber/bibtex for article citation under publications.
% 2. Figure out a smoother way for the document to flow onto the next page.
% 3. Add styling information for a "Projects/Hacks" section.
% 4. Add location/address information
% 5. Merge OpenFont and MacFonts as a single sty with options.
% 
%%%%%%%%%%%%%%%%%%%%%%%%%%%%%%%%%%%%%%
%
% CHANGELOG:
% v1.1:
% 1. Fixed several compilation bugs with \renewcommand
% 2. Got Open-source fonts (Windows/Linux support)
% 3. Added Last Updated
% 4. Move Title styling into .sty
% 5. Commented .sty file.
%
%%%%%%%%%%%%%%%%%%%%%%%%%%%%%%%%%%%%%%%
%
% Known Issues:
% 1. Overflows onto second page if any column's contents are more than the
% vertical limit
% 2. Hacky space on the first bullet point on the second column.
%
%%%%%%%%%%%%%%%%%%%%%%%%%%%%%%%%%%%%%%


\documentclass[]{deedy-resume-openfont}
\usepackage{fancyhdr}
    
\pagestyle{fancy}
\fancyhf{}
    
\begin{document}

%%%%%%%%%%%%%%%%%%%%%%%%%%%%%%%%%%%%%%
%
%     LAST UPDATED DATE
%
%%%%%%%%%%%%%%%%%%%%%%%%%%%%%%%%%%%%%%
\lastupdated

%%%%%%%%%%%%%%%%%%%%%%%%%%%%%%%%%%%%%%
%
%     TITLE NAME
%
%%%%%%%%%%%%%%%%%%%%%%%%%%%%%%%%%%%%%%
\namesection{孙}{海洲}{ \urlstyle{same}\href{haizhou.uestc2011@gmail.com}{haizhou.uestc2011@gmail.com} | 1851 0343 050 | {寻找 2019 SDE/SRE/AI infra 全职工作}
}

%%%%%%%%%%%%%%%%%%%%%%%%%%%%%%%%%%%%%%
%
%     COLUMN ONE
%
%%%%%%%%%%%%%%%%%%%%%%%%%%%%%%%%%%%%%%

\begin{minipage}[t]{0.3\textwidth} 

%%%%%%%%%%%%%%%%%%%%%%%%%%%%%%%%%%%%%%
%     EDUCATION
%%%%%%%%%%%%%%%%%%%%%%%%%%%%%%%%%%%%%%

\section{教育经历} 
\sectionsep

\subsection{中国科学院大学}
\descript{硕士学位,计算机技术}
\location{2017.09-2019.03}
\sectionsep

\subsection{电子科技大学}
\descript{学士学位,软件工程}
\location{2011.09-2015.06}
\sectionsep

%%%%%%%%%%%%%%%%%%%%%%%%%%%%%%%%%%%%%%
%     LINKS
%%%%%%%%%%%%%%%%%%%%%%%%%%%%%%%%%%%%%%

\section{链接}
\sectionsep    
Github:// \href{https://github.com/haiker2011}{haiker2011} \\
{(\textbf{ 20+ }关注者)} \\

\section{部分博文}
\sectionsep
\href{http://gaocegege.com/Blog/kubernetes/operator}{Kubernetes CRD Operator 实现指南} \\
\href{http://gaocegege.com/Blog/%E6%9C%BA%E5%99%A8%E5%AD%A6%E4%B9%A0/katib}{Katib: Kubernetes native 的超参数训练系统} \\
\href{http://gaocegege.com/Blog/ml%20system/kubeflow}{Kubeflow: 在 Kubernetes 上进行机器学习} \\
\href{http://gaocegege.com/Blog/%E6%BA%90%E7%A0%81%E5%88%86%E6%9E%90/kubernetes-scheduler}{浅入了解容器编排框架调度器之 Kubernetes} \\
\href{http://gaocegege.com/Blog/%E9%9A%8F%E7%AC%94/consistency}{小议分布式系统的一致性模型} \\
\sectionsep

\section{语言能力}
\sectionsep
英语四级 531 \\
\sectionsep

%%%%%%%%%%%%%%%%%%%%%%%%%%%%%%%%%%%%%%
%     COURSEWORK
%%%%%%%%%%%%%%%%%%%%%%%%%%%%%%%%%%%%%%

% \section{修读课程}
% \subsection{Graduate}
% Advanced Machine Learning \\
% Open Source Software Engineering \\
% Advanced Interactive Graphics \\
% Compilers + Practicum \\
% Cloud Computing \\
% Evolutionary Computation \\
% Defending Computer Networks \\
% Machine Learning \\
% \sectionsep



%%%%%%%%%%%%%%%%%%%%%%%%%%%%%%%%%%%%%%
%     EXPERIENCE
%%%%%%%%%%%%%%%%%%%%%%%%%%%%%%%%%%%%%%

\section{工作实习经历}

\sectionsep
\runsubsection{北京数字认证股份有限公司}
\descript{Android 开发工程师}
\location{2015.09-2017.07 | 北京}
\vspace{\topsep}
\vspace{\topsep}
\begin{tightemize}
\item Android API 开发
\end{tightemize}
\sectionsep

\sectionsep
\runsubsection{成都尚医信息科技有限公司}
\descript{Android 开发工程师(实习)}
\location{2015.02-2015.07 | 成都}
\vspace{\topsep}
\begin{tightemize}
\item 术康 Android 端开发
\end{tightemize}
\sectionsep

\sectionsep
\runsubsection{长虹}
\descript{Android 开发工程师(实习)}
\location{2013.03-2013.07 | 绵阳}
\vspace{\topsep}
\begin{tightemize}
\item 实现智能灯控系统 Android 端
\end{tightemize}
\sectionsep

%%%%%%%%%%%%%%%%%%%%%%%%%%%%%%%%%%%%%%
%     RESEARCH
%%%%%%%%%%%%%%%%%%%%%%%%%%%%%%%%%%%%%%

\section{项目与论文}
\sectionsep

\runsubsection{\href{https://github.com/gaocegege/scrala}{\bf Scrala}}
\descript{Owner}
\location{2015.12}
\begin{tightemize}
    \item 使用 scala 实现的爬虫框架,灵感来自 scrapy
    \item 在 GitHub 上获得 \textbf{70 stars}
    \item 底层使用 Actor 模型取代 Python 中的异步模型
    \end{tightemize}
\sectionsep
\end{minipage}

\newpage
\pagestyle{fancy}
\fancyhf{}

% PAGE TWO

\begin{minipage}[t]{0.3\textwidth}

%%%%%%%%%%%%%%%%%%%%%%%%%%%%%%%%%%%%%%
%     SKILLS
%%%%%%%%%%%%%%%%%%%%%%%%%%%%%%%%%%%%%%

\section{技能}
\sectionsep
\subsection{编程}
\location{超过 5000 行}
Java \textbullet{} Python  \\
\location{1000 - 5000 行}
Go \textbullet{} C \textbullet{} C++ \textbullet{} Shell \\
\location{低于 1000 行}
Elixir \textbullet{} Javascript \textbullet{} \LaTeX\ \\ 
\sectionsep

\subsection{云计算}
\location{一般}
Docker \textbullet{} Kubernetes \textbullet{} Istio \textbullet{} Knative \\
\location{了解}
Docker \textbullet{} Kubernetes  \\
\sectionsep

\subsection{DevOps}
\location{一般}
微服务 \textbullet{} 云原生开发 \textbullet{} Jenkins \textbullet{} Travis CI

%%%%%%%%%%%%%%%%%%%%%%%%%%%%%%%%%%%%%%
%
%     COLUMN TWO
%
%%%%%%%%%%%%%%%%%%%%%%%%%%%%%%%%%%%%%%

\end{minipage} 
\hfill
\begin{minipage}[t]{0.68\textwidth} 


%%%%%%%%%%%%%%%%%%%%%%%%%%%%%%%%%%%%%%
%     OPEN SOURCE
%%%%%%%%%%%%%%%%%%%%%%%%%%%%%%%%%%%%%%

%%%%%%%%%%%%%%%%%%%%%%%%%%%%%%%%%%%%%%
%     AWARDS
%%%%%%%%%%%%%%%%%%%%%%%%%%%%%%%%%%%%%%

\runsubsection{\href{https://github.com/haiker2011/awesome-nlp-sentiment-analysis}{awesome-nlp-sentiment-analysis}}
\descript{Owner}
\location{2018.11}
\begin{tightemize}
    \item 最新情感分析论文整理,在 GitHub 上获得 \textbf{10 stars}
    \item 论文、代码、数据集方面大家研究
    \end{tightemize}
\sectionsep

%%%%%%%%%%%%%%%%%%%%%%%%%%%%%%%%%%%%%%
%     PUBLICATIONS
%%%%%%%%%%%%%%%%%%%%%%%%%%%%%%%%%%%%%%

% \section{Publications} 
% \renewcommand\refname{\vskip -1.5cm} % Couldn't get this working from the .cls file
% \bibliographystyle{abbrv}
% \bibliography{publications}
% \nocite{*}


%%%%%%%%%%%%%%%%%%%%%%%%%%%%%%%%%%%%%%
%     AWARDS
%%%%%%%%%%%%%%%%%%%%%%%%%%%%%%%%%%%%%%

\section{所获奖项} 
\begin{tabular}{rll}
2017         & 奖学金  & 因特尔奖学金 \\

\end{tabular}
\sectionsep


%%%%%%%%%%%%%%%%%%%%%%%%%%%%%%%%%%%%%%
%     OPEN SOURCE
%%%%%%%%%%%%%%%%%%%%%%%%%%%%%%%%%%%%%%

\section{开源贡献}
\begin{tabular}{ll}
\href{https://github.com/haiker2011/awesome-nlp-sentiment-analysis/commits?author=haiker2011}{\bf haiker2011/awesome-nlp-sentiment-analysis} & 情感分析论文整理,项目 maintainer \\
\href{https://github.com/servicemesher/getting-started-with-knative/commits?author=haiker2011}{\bf servicemesher/getting-started-with-knative} & 《Knative 入门中文版》第三章翻译 \\
\href{https://github.com/rootsongjc/awesome-cloud-native/commits?author=haiker2011}{\bf rootsongjc/awesome-cloud-native} & 添加AI模块相关云原生组件 \\
\href{https://github.com/servicemesher/trans/commits?author=haiker2011}{\bf servicemesher/trans} & 翻译文章 Knative:精简代码之道 \\
\end{tabular}
\sectionsep





\end{minipage} 
\end{document}  \documentclass[]{article}
